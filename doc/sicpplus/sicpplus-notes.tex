% !TEX TS-program = arara
% arara: xelatex: { synctex: on, options: [-halt-on-error] } 
% arara: biber
% % arara: texindy: { markup: xelatex }
% %% arara: makeglossaries
% % arara: xelatex: { synctex: on, options: [-interaction=batchmode, -halt-on-error] }
% % arara: xelatex: { synctex: on, options: [-interaction=batchmode, -halt-on-error]  }
% % arara: clean: { extensions: [ aux, log, out, run.xml, ptc, toc, mw, synctex.gz, ] }
% % arara: clean: { extensions: [ bbl, bcf, blg, ] }
% % arara: clean: { extensions: [ glg, glo, gls, ] }
% % arara: clean: { extensions: [ idx, ilg, ind, xdy, ] }
% % arara: clean: { extensions: [ plCode, plData, plMath, plExercise, plNote, plQuote, ] }
%-----------------------------------------------------------------
\documentclass[11pt]{PalisadesLakesBook}
% \geomHDTV
% \geomLandscape
\geomHalfDTV
\geomPortraitOneColumn
%-----------------------------------------------------------------

%\AsanaFonts % misssing \mathhyphen; less on page than Cormorant/Garamond
%\CormorantFonts % light, missing unicode greek
\EBGaramondFonts % fewest pages
%\ErewhonFonts
%\FiraFonts % tall lines, all sans, much less per page, missing \in?
%\GFSNeohellenicFonts 
%\KpFonts
%\LatinModernFonts
%\LegibleFonts
%\LibertinusFonts
%\NewComputerModernFonts
%\STIXTwoFonts
%\BonumFonts % most pages
%\PagellaFonts
%\ScholaFonts
%\TermesFonts
%\XITSFonts

%-----------------------------------------------------------------
\togglefalse{plMath}
\togglefalse{plCode}
\togglefalse{plData}
\togglefalse{plNote}
\togglefalse{plExercise}
\togglefalse{plQuote}
\togglefalse{printglossary}
\togglefalse{printindex}
%-----------------------------------------------------------------
\title{Notes on\\
 SICP,\\
  SICM,\\
   Functional Differential Geometry,\\ 
 Software Design for Flexibility, \\
 etc.}
\author{John Alan McDonald 
(palisades dot lakes at gmail dot com)}
\date{draft of \today}
%-----------------------------------------------------------------
\begin{document}
\maketitle
\PalisadesLakesTableOfContents{7}
%-----------------------------------------------------------------
\def\sharedFolder{../../shared/}
%-----------------------------------------------------------------
\begin{plSection}{Introduction}

Various works of Abelson, Sussman, et al.
 
Partially notes on reading; 
partially my own work-in-progress analysis 
and implementation in 
Clojure~\cite{
EmerickCarperGrand:2012:ClojureProgramming,
FogusHouser:2011:JoyOfClojure,
Halloway:2009:Clojure,
Hickey:2012:Clojure,
Rathore:2011:Clojure,
VanderhartSierra:2009:Clojure,}.

\citeAuthorYearTitle{HalfantSussman:1987:AbstractNumerical}

\citeAuthorYearTitle{AbelsonSussman:1996:SICP}

\citeAuthorYearTitle{SussmanWisdom:2013:FunctionalDifferentialGeometry}

\citeAuthorYearTitle{SussmanWisdom:2015:SICM2}

\citeAuthorYearTitle{HansonSussman:2021:SDFF}

%-----------------------------------------------------------------
\end{plSection}%{Introduction}
%-----------------------------------------------------------------
\begin{plSection}{Reading}
%-----------------------------------------------------------------
\begin{plSection}{\citeAuthorYearTitle{HansonSussman:2021:SDFF}}
%-----------------------------------------------------------------
\begin{plSection}{Forward (Guy Steele)}

Describes book as \emph{master class}, not tutorial.

Argues for bundling code plus data as \emph{closures}. 
Also \emph{generic functions}.

Not, perhaps, the clearest such argument, but how much can you do
in a 2 pages forward?

My argument would go something like:
Data is just bits, meaningless on its own.
The code that manipulates those bits determines their 
\emph{meaning}.
The farther the ``distance'' (in some sense) 
between code an data, 
the higher the likelihood of confusion,
of misinterpreting bits. 

\end{plSection}%
{Forward (Guy Steele)}
%-----------------------------------------------------------------
\begin{plSection}{Preface}

One goal is to modify by addition, 
without changing existing code.

Begins talking about a very questionable analogy 
to biological systems.
Possibly a fallacy inherited from traditional AI.
Particularly ironic in the middle of a pandemic.

Something I believe:
\begin{plQuote}
{\citeAuthorYearTitle[p~xiii]{HansonSussman:2021:SDFF}}{}
We have often programmed ourselves into corners
and had to expend great effort to escape from those corners.
We have accumulated enough experience to feel that we can
identify, isolate, and demonstrate strategies
that we have found to be effective for building large systems
that can be adapted for purposes 
that were not anticipated in the original design. {\ldots}
\end{plQuote}
However, they may be over promising in some sense.

Sometimes, the best choice is starting over.
There are different levels of ``starting over'':
for example, complete re-design, 
re-implementing a mostly-the-same API.

My own experience is that, when having a choice
between starting from scratch and fixing the existing code,
the times I chose to start over were never a mistake,
and many, if not most, of the times I chose to ``fix'' existing
code turned out to cost more time than starting over would have.
I think I am more likely than most to advocate starting clean,
and in my experience I've been too conservative in that direction.
The cause may be over-estimating the cost of something new
relative to the costs of maintaining/updating something that
already exists. 

Plus fear of the unknown: legacy code whose authors have vanished,
and no one present fully understands what it is doing or why.
The fear is that there is some lost lesson in the existing
code that will have to be painfully relearned in anything new.

We will see if the book has any to say about making such choices.

\end{plSection}%{Preface}
%-----------------------------------------------------------------
\begin{plSection}{Acknowledgements}
\begin{plQuote}
{\citeAuthorYearTitle[p~xiix]{HansonSussman:2021:SDFF}}{}
In many ways this book is an advanced sequel to SICP.
\end{plQuote}

Sussman acknowledges discussions (in graduate school?)
with Minsky and fellow students---possibly a source for the
biologically motivated old-style AI ideas that appear.
\end{plSection}%{Acknowledgements}
%-----------------------------------------------------------------
\begin{plSection}{Ch 1: Flexibility in nature and design}

I find this chapter disappointing.

I think the fundamental flaw is using 
inappropriate analogies to (physical building) architecture
and, especially, evolution, 
to justify their ideas about software artifact design
and construction,
rather than arguing on first principles.
This is some version of the ``false analogy'' 
fallacy~\cite{wiki:ArgumentFromAnalogyFalse},
or perhaps the
``appeal to authority'' fallacy~\cite{wiki:ArgumentFromAuthority},
where the ``authority'' isn't talking about the problem at hand, 
and is not so clearly an ``authority'' in their own domain either.

Another issue is that they seem to talk about 
specialized vs general purpose
and success vs failure as both binary.
Any tool will have some domain of tasks where it can be used
with varying degrees of success.
Possibly the argument here should really be about 
how (and when or if) making things more general increases
the probable degree of success. 

I think they may be missing the need for humility in specifying
what a piece of software (or any tool) will be used for.
Enlarging the domain of applicability, in the right ways,
increases the chances that it will be useful when circumstances 
change, and, more important, 
when the first real use exposes all sorts of things 
you simply failed to think of ahead of time
(\textbf{TODO:} an example would be nice here.)

%-----------------------------------------------------------------
\begin{plSection}{Additive programming}

Not crazy about the name.

Code can be flexible if it permits updates by \emph{replacing}
parts, not just \emph{adding} parts.
Common case: 
replace old implementation with (almost) equivalent 
higher performance one.

In either case, proper abstraction makes it less likely
an update will break something that currently works.

Advocates robustness in the sense of
(1) accepting large domain of inputs
and (2) producing a small ``range of outputs of a part'' 
(codomain).

Not sure I believe this.
Depends on what ``accepting'' means in (1).
Often the right choice is to clearly reject unacceptable inputs
as soon as possible.
For example, if somebody asks for $\sqrt{\text{blue}}$
you generally don't want to quietly return \texttt{0x00000F}.

\end{plSection}%{Additive programming}
%-----------------------------------------------------------------

\begin{plQuote}
{\citeAuthorYearTitle[p~xiix]{HansonSussman:2021:SDFF}}{}
\end{plQuote}


\end{plSection}%{Ch 1: Flexibility in nature and design}
%-----------------------------------------------------------------
\end{plSection}%{\citeAuthorYearTitle{HansonSussman:2021:SDFF}}
%-----------------------------------------------------------------
\end{plSection}%{Reading}
%-----------------------------------------------------------------
\begin{plSection}{Implementation}
I am stretching the work ``implementation'' to cover both
software and a mathematical formulation of the problems.
%-----------------------------------------------------------------
\begin{plSection}{Foundation sketch}

Idea: base mathematics on an ideal (Turing-like) machine.
Then mapping to usable software is more-or-less mapping
the ideal machine to a realizable one.

Ideal machine: 
\begin{itemize}
  \item Turing machine~\cite{
  Turing:1936:Computability,
  Turing:1937:ComputabilityLambda,
  Turing:1938:ComputabilityCorrection},
  Lambda calculus, etc.,
  equivalent,
  but more ``convenient''.
  \item unbounded (but not infinite)
  \item deterministic (?)
  \item references with tagged values
  \item write-once memory (but maybe that's not ``convenient'')?
  \item Primitive values plus data structures built out of tagged
  references.
\end{itemize}

Ideal (formal) language. Lambda calculus but more ``convenient''.

Math based on procedures (procedural functions) 
in the ideal language.

Everything must handle the possible of non-terminating procedures
(on some or all inputs).

A set is a procedure that returns true or false for any input
value.
As opposed to axiomatic set theory.
I think this eliminates the set paradoxes
that led to the ``crisis in mathematics'' circa 1900~\cite{
Feferman:2000:ConstructivePredicativeClassicalAnalysis,
Ferreiros:2008:Crisis}.

A proof is a procedure that takes a set of axiom expressions
and returns another (according to some rules).

End result is something like constructive analysis
with no more than countable infinity, if that.
The key spaces in the readings will be built out of 
some version of computable reals.

See:

\citeAuthorYearTitle{Feferman:1989:IsCantorNecessary}

\citeAuthorYearTitle[ch.~2 ``Is Cantor necessary?'']{Feferman:1998:LightOfLogic}

\citeAuthorYearTitle[ch.~12 ``Is Cantor necessary? (Conclusion)'']{Feferman:1998:LightOfLogic}

\citeAuthorYearTitle{Feferman:1992:ALittleBit}

\citeAuthorYearTitle[ch~14 ``Why a little bit goes a long way:
logical foundations of scientifically applicable mathematics'']{Feferman:1998:LightOfLogic}


\end{plSection}%{Foundation sketch}
%-----------------------------------------------------------------
\end{plSection}%{Implementation}
%-----------------------------------------------------------------
\BeginAppendices
%-----------------------------------------------------------------
\begin{plSection}{Typesetting this document}

This document was typeset using Mik\TeX{} $21.3$ 
\cite{Schenk:2017:Miktex} 
and {\TeX}works $0.6.5$ \cite{KewLoffler:2017:Texworks} 
on \textsc{Windows} $10$. 
I used \texttt{arara} \cite{CeredaEtAl:2021:Arara} 
to run \texttt{xelatex}, \texttt{biber}, \texttt{makeglossaries},  
and
\texttt{texindy: { markup: xelatex }}.
I believe only Mik\TeX\  and {\TeX}works are Windows specific; 
the actual typesetting tools should be usable on Linux and MacOS 
as well.

See also \cite{Talbot:2012:LatexNovices,Talbot:2013:LatexPhD}.

\begin{plScreen}
{Configuring {\TeX}works for \texttt{arara}.}
{fig:arara}
\centering
\includegraphics[scale=0.75]{arara.png}
\end{plScreen}

As of 2021-03-16, I am defaulting to 'upright' math text.
I am considering trying sans symbols (sum, integral) 
as the default as well,
but I'm not sure if the existing sans math fonts are as well
developed as the others, especially for non-letter symbols.

Goals are:
\begin{enumerate}
\item Readability (at least subjectively). Prefer top-down over
    bottom-up understanding. In particular, it should be easy to
    pick out the general structure of an expression, ignoring
    less important details.

\item General appearance should be something like regularized,
    cleaned up, good handwriting. To me this suggests upright
    sans letters and compatible symbols.

\item Math expressions should be visually distinct from regular
    text, especially when mixed together --- but not so much
    that it disturbs the flow of reading. To me, this suggests
    serif fonts for text and, again, sans for math, upright
    for both, and restraint in weight variation.
\end{enumerate}

I am aware of ISO 80000-2:2019, but haven't read it,
since the price is about US\$170 and I doubt I would agree with
their choices. Indirect sources suggest that in ISO 80000-2
most symbols are set in italic, which I find difficult to read,
and believe should be restricted, if used at all, to placing
emphasis on very short pieces of text, on the order of a word
or short phrase.~\cite{wiki:TypoConventionsMath}
And even in that case, and contrary to most
advice I've seen, I prefer slanted faces to italic
(one agreeing opinion, for text at least: 
\citeAuthorYearTitle{RoddTornqkvist:2009:LMSStyleGuide}.
My subjective feeling is that italic text,
in most fonts, is read more-or-less single letter at a time,
rather than word or phrase at a time. I think the same applies
to mathematical expressions, forcing the reader work bottom up,
to first parse out individual letters/symbols and then
consciously re-assemble them, rather than grasping
the expression as a whole, and then acquiring details top-down.


\vfill

\end{plSection}%{Typesetting}

%-----------------------------------------------------------------
%-----------------------------------------------------------------
\end{document}
%-----------------------------------------------------------------
