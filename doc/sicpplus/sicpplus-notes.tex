% !TEX TS-program = arara
% arara: xelatex: { synctex: on, options: [-halt-on-error] } 
% arara: biber
% % arara: texindy: { markup: xelatex }
% %% arara: makeglossaries
% % arara: xelatex: { synctex: on, options: [-interaction=batchmode, -halt-on-error] }
% % arara: xelatex: { synctex: on, options: [-interaction=batchmode, -halt-on-error]  }
% % arara: clean: { extensions: [ aux, log, out, run.xml, ptc, toc, mw, synctex.gz, ] }
% % arara: clean: { extensions: [ bbl, bcf, blg, ] }
% % arara: clean: { extensions: [ glg, glo, gls, ] }
% % arara: clean: { extensions: [ idx, ilg, ind, xdy, ] }
% % arara: clean: { extensions: [ plCode, plData, plMath, plExercise, plNote, plQuote, ] }
%-----------------------------------------------------------------
\documentclass[11pt]{PalisadesLakesBook}
% \geomHDTV
% \geomLandscape
\geomHalfDTV
\geomPortraitOneColumn
%-----------------------------------------------------------------

%\AsanaFonts % misssing \mathhyphen; less on page than Cormorant/Garamond
%\CormorantFonts % light, missing unicode greek
\EBGaramondFonts % fewest pages
%\ErewhonFonts
%\FiraFonts % tall lines, all sans, much less per page, missing \in?
%\GFSNeohellenicFonts 
%\KpFonts
%\LatinModernFonts
%\LegibleFonts
%\LibertinusFonts
%\NewComputerModernFonts
%\STIXTwoFonts
%\BonumFonts % most pages
%\PagellaFonts
%\ScholaFonts
%\TermesFonts
%\XITSFonts

%-----------------------------------------------------------------
\togglefalse{plMath}
\togglefalse{plCode}
\togglefalse{plData}
\togglefalse{plNote}
\togglefalse{plExercise}
\togglefalse{plQuote}
\togglefalse{printglossary}
\togglefalse{printindex}
%-----------------------------------------------------------------
\title{Notes on SICP, SICM, Functional Differential Geometry, 
and Software Design for Flexibility, etc.}
\author{John Alan McDonald 
(palisades dot lakes at gmail dot com)}
\date{draft of \today}
%-----------------------------------------------------------------
\begin{document}
\maketitle
\PalisadesLakesTableOfContents{7}
%-----------------------------------------------------------------
\def\sharedFolder{../../shared/}
%-----------------------------------------------------------------
\begin{plSection}{Introduction}

Various works of Abelson, Sussman, et al.
 
Partially notes on reading; 
partially my own work-in-progress analysis 
and implementation in 
Clojure~\cite{
EmerickCarperGrand:2012:ClojureProgramming,
FogusHouser:2011:JoyOfClojure,
Halloway:2009:Clojure,
Hickey:2012:Clojure,
Rathore:2011:Clojure,
VanderhartSierra:2009:Clojure,}.

\citeAuthorYearTitle{HalfantSussman:1987:AbstractNumerical}

\citeAuthorYearTitle{AbelsonSussman:1996:SICP}

\citeAuthorYearTitle{SussmanWisdom:2013:FunctionalDifferentialGeometry}

\citeAuthorYearTitle{SussmanWisdom:2015:SICM2}

\citeAuthorYearTitle{HansonSussman:2021:SDFF}

%-----------------------------------------------------------------
\end{plSection}%{Introduction}
%-----------------------------------------------------------------
\begin{plSection}{Reading}
%-----------------------------------------------------------------
\begin{plSection}{\citeAuthorYearTitle{HansonSussman:2021:SDFF}}
%-----------------------------------------------------------------
\begin{plSection}{Forward (Guy Steele)}
\end{plSection}{Forward (Guy Steele)}
%-----------------------------------------------------------------
\begin{plSection}{Preface}
\end{plSection}{Preface}
%-----------------------------------------------------------------
\end{plSection}%{\citeAuthorYearTitle{HansonSussman:2021:SDFF}}
%-----------------------------------------------------------------
\end{plSection}%{Reading}
%-----------------------------------------------------------------
\begin{plSection}{Implementation}
I am stretching the work ``implementation'' to cover both
software and a mathematical formulation of the problems.
%-----------------------------------------------------------------
\begin{plSection}{Foundation sketch}

Idea: base mathematics on an ideal (Turing-like) machine.
Then mapping to usable software is more-or-less mapping
the ideal machine to a realizable one.

Ideal machine: 
\begin{itemize}
  \item Turing machine~\cite{
  Turing:1936:Computability,
  Turing:1937:ComputabilityLambda,
  Turing:1938:ComputabilityCorrection},
  Lambda calculus, etc.,
  equivalent,
  but more ``convenient''.
  \item unbounded (but not infinite)
  \item deterministic (?)
  \item references with tagged values
  \item write-once memory (but maybe that's not ``convenient'')?
  \item Primitive values plus data structures built out of tagged
  references.
\end{itemize}

Ideal (formal) language. Lambda calculus but more ``convenient''.

Math based on procedures (procedural functions) 
in the ideal language.

Everything must handle the possible of non-terminating procedures
(on some or all inputs).

A set is a procedure that returns true or false for any input
value.
As opposed to axiomatic set theory.
I think this eliminates the set paradoxes
that led to the ``crisis in mathematics'' circa 1900~\cite{
Feferman:2000:ConstructivePredicativeClassicalAnalysis,
Ferreiros:2008:Crisis}.


A proof is a procedure that takes a set of axiom expressions
and returns another (according to some rules).

End result is something like constructive analysis
with no more than countable infinity, if that.
The key spaces in the readings will be built out of 
some version of computable reals.

See:

\citeAuthorYearTitle{Feferman:1989:IsCantorNecessary}

\citeAuthorYearTitle[ch.~2 ``Is Cantor necessary?'']{Feferman:1998:LightOfLogic}

\citeAuthorYearTitle[ch.~12 ``Is Cantor necessary? (Conclusion)'']{Feferman:1998:LightOfLogic}

\citeAuthorYearTitle{Feferman:1992:ALittleBit}

\citeAuthorYearTitle[ch~14 ``Why a little bit goes a long way:
logical foundations of scientifically applicable mathematics'']{Feferman:1998:LightOfLogic}


\end{plSection}%{Foundation sketch}
%-----------------------------------------------------------------
\end{plSection}%{Implementation}
%-----------------------------------------------------------------
\BeginAppendices
%-----------------------------------------------------------------
\begin{plSection}{Typesetting this document}

This document was typeset using Mik\TeX{} $21.3$ 
\cite{Schenk:2017:Miktex} 
and {\TeX}works $0.6.5$ \cite{KewLoffler:2017:Texworks} 
on \textsc{Windows} $10$. 
I used \texttt{arara} \cite{CeredaEtAl:2021:Arara} 
to run \texttt{xelatex}, \texttt{biber}, \texttt{makeglossaries},  
and
\texttt{texindy: { markup: xelatex }}.
I believe only Mik\TeX\  and {\TeX}works are Windows specific; 
the actual typesetting tools should be usable on Linux and MacOS 
as well.

See also \cite{Talbot:2012:LatexNovices,Talbot:2013:LatexPhD}.

\begin{plScreen}
{Configuring {\TeX}works for \texttt{arara}.}
{fig:arara}
\centering
\includegraphics[scale=0.75]{arara.png}
\end{plScreen}

As of 2021-03-16, I am defaulting to 'upright' math text.
I am considering trying sans symbols (sum, integral) 
as the default as well,
but I'm not sure if the existing sans math fonts are as well
developed as the others, especially for non-letter symbols.

Goals are:
\begin{enumerate}
\item Readability (at least subjectively). Prefer top-down over
    bottom-up understanding. In particular, it should be easy to
    pick out the general structure of an expression, ignoring
    less important details.

\item General appearance should be something like regularized,
    cleaned up, good handwriting. To me this suggests upright
    sans letters and compatible symbols.

\item Math expressions should be visually distinct from regular
    text, especially when mixed together --- but not so much
    that it disturbs the flow of reading. To me, this suggests
    serif fonts for text and, again, sans for math, upright
    for both, and restraint in weight variation.
\end{enumerate}

I am aware of ISO 80000-2:2019, but haven't read it,
since the price is about US\$170 and I doubt I would agree with
their choices. Indirect sources suggest that in ISO 80000-2
most symbols are set in italic, which I find difficult to read,
and believe should be restricted, if used at all, to placing
emphasis on very short pieces of text, on the order of a word
or short phrase.~\cite{wiki:TypoConventionsMath}
And even in that case, and contrary to most
advice I've seen, I prefer slanted faces to italic
(one agreeing opinion, for text at least: 
\citeAuthorYearTitle{RoddTornqkvist:2009:LMSStyleGuide}.
My subjective feeling is that italic text,
in most fonts, is read more-or-less single letter at a time,
rather than word or phrase at a time. I think the same applies
to mathematical expressions, forcing the reader work bottom up,
to first parse out individual letters/symbols and then
consciously re-assemble them, rather than grasping
the expression as a whole, and then acquiring details top-down.


\vfill

\end{plSection}%{Typesetting}

%-----------------------------------------------------------------
%-----------------------------------------------------------------
\end{document}
%-----------------------------------------------------------------
